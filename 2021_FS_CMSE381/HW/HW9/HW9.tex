% Don't touch this %%%%%%%%%%%%%%%%%%%%%%%%%%%%%%%%%%%%%%%%%%%
\documentclass[11pt]{article}
\usepackage{fullpage}
\usepackage[left=1in,top=1in,right=1in,bottom=1in,headheight=3ex,headsep=3ex]{geometry}
\usepackage{graphicx}
\usepackage{float}


\newtheorem{theorem}{Theorem}
\newtheorem{lemma}[theorem]{Lemma}
\newtheorem{proposition}[theorem]{Proposition}
\newtheorem{corollary}[theorem]{Corollary}
\newtheorem{fact}[theorem]{Fact}
\newtheorem{definition}[theorem]{Definition}
\newtheorem{remark}[theorem]{Remark}
\newtheorem{conjecture}[theorem]{Conjecture}
\newtheorem{question}[theorem]{Question}
\newtheorem{answer}[theorem]{Answer}
\newtheorem{exercise}[theorem]{Exercise}
\newtheorem{example}[theorem]{Example}
\newenvironment{proof}{\noindent \textbf{Proof:}}{$\Box$}


\newcommand{\blankline}{\quad\pagebreak[2]}
%%%%%%%%%%%%%%%%%%%%%%%%%%%%%%%%%%%%%%%%%%%%%%%%%%%%%%%%%%%%%%

% Modify Course title, instructor name, semester here %%%%%%%%

\title{\underline{CMSE 381: HW9}}
\date{}

%%%%%%%%%%%%%%%%%%%%%%%%%%%%%%%%%%%%%%%%%%%%%%%%%%%%%%%%%%%%%%

% Don't touch this %%%%%%%%%%%%%%%%%%%%%%%%%%%%%%%%%%%%%%%%%%%
\usepackage[sc]{mathpazo}
\linespread{1.05} % Palatino needs more leading (space between lines)
\usepackage[T1]{fontenc}
\usepackage[mmddyyyy]{datetime}% http://ctan.org/pkg/datetime
\usepackage{advdate}% http://ctan.org/pkg/advdate
\newdateformat{syldate}{\twodigit{\THEMONTH}/\twodigit{\THEDAY}}
\newsavebox{\MONDAY}\savebox{\MONDAY}{Mon}% Mon
\newcommand{\week}[2]{%
%  \cleardate{mydate}% Clear date
% \newdate{mydate}{\the\day}{\the\month}{\the\year}% Store date
  \paragraph*{\kern-2ex\quad #1, \syldate{\today} - \AdvanceDate[4]\syldate{\today}:}% Set heading  \quad #1
%  \setbox1=\hbox{\shortdayofweekname{\getdateday{mydate}}{\getdatemonth{mydate}}{\getdateyear{mydate}}}%
  \ifdim\wd1=\wd\MONDAY
    \AdvanceDate[7]
  \else
    \AdvanceDate[7]
  \fi%
}
\usepackage{setspace}
\usepackage{multicol}
%\usepackage{indentfirst}
\usepackage{fancyhdr,lastpage}
\usepackage{url}
\pagestyle{fancy}
\usepackage{hyperref}
\usepackage{lastpage}
\usepackage{amsmath}
\usepackage{layout}

\lhead{}
\chead{}
%%%%%%%%%%%%%%%%%%%%%%%%%%%%%%%%%%%%%%%%%%%%%%%%%%%%%%%%%%%%%%


%%%%%%%%%%%%%%%%%%%%%%%%%%%%%%%%%%%%%%%%%%%%%%%%%%%%%%%%%%%%%%
% Don't touch this %%%%%%%%%%%%%%%%%%%%%%%%%%%%%%%%%%%%%%%%%%%
\lfoot{}
\cfoot{\small \thepage/\pageref*{LastPage}}
\rfoot{}

\usepackage{array, xcolor}
\usepackage{color,hyperref}
\definecolor{clemsonorange}{HTML}{EA6A20}
\hypersetup{colorlinks,breaklinks,linkcolor=clemsonorange,urlcolor=clemsonorange,anchorcolor=clemsonorange,citecolor=black}
\usepackage{txfonts}
\begin{document}

\maketitle

\blankline

\begin{enumerate}
	\item[1] (30 pts) Exercise 7.9.1
	\item[2] (10 pts) Exercise 7.9.2 
	\item[3] (10 pts) Exercise 7.9.3
	\item[4] (10 pts) Exercise 7.9.4
	\item[5] (20 pts) Exercise 7.9.9
	\item[6] (20 pts) Exercise 7.9.11
	\item[7] (Challenging problem) \textit{Derivation of smoothing splines} Suppose that $N \geq 2$, and that $g$ is the natural cubic spline interpolant to the pairs $\{x_i, z_i \}_{i=1}^N$, with $a < x_1 < \cdot < x_N < b$. This is a natural spline with a knot at every $x_i$; being an $N-$dimensional space of functions, we can determine the coefficients such that it interpolates the sequence $z_i$ exactly.
Let $\tilde{g}$ be any other differentiable function on $[a, b]$ that interpolates the $N$ pairs. 
\begin{itemize}
	\item[(a).] Let $h(x) = \tilde{g}(x) - g(x)$. Use integration by parts and the fact that $g$ is a natural cubic spline to show that 
	\[
	\int_a^b g''(x) h''(x) dx = - \sum_{j =1}^{N - 1} g'''(x^+_j) \{ h(x_{j + 1} - h(x_j) \} = 0 
	\]
	\item[(b).] Hence show that 
	\[
		\int_a^b \tilde{g}''(t)^2 dt  \geq \int_a^b g''(t)^2 dt,
	\]
	and that equality can only hold if $h$ is identically zero in $[a, b]$.
	\item[(c).] Consider the penalized least squares problem
	\[
	\min_f \left[ \sum_{i= 1}^N (y_i - f(x_i))^2 + \lambda \int_a^b f''(t)^2 dt  \right].
	\]
	Use (b) to argue that the minimizer must be a cubic spline with knots at each of the $x_i$.
\end{itemize}


\end{enumerate}

\end{document}



